%Wir verwenden eine DIN-A4-Seite und die Schriftgröße 12.
\documentclass[a4paper,12pt]{scrartcl} 


%Diese drei Pakete benötigen wir für die Umlaute, Deutsche Silbentrennung etc.
%Apple-Nutzer sollten anstelle von \usepackage[latin1]{inputenc} das Paket \usepackage[applemac]{inputenc} verwenden
%% \usepackage[latin1]{inputenc}
%%apt-get install texlive-lang-german damit ngerman keine Probleme mehr macht !!
\usepackage[utf8]{inputenc} 
\usepackage[T1]{fontenc}
\usepackage[ngerman]{babel}

%Das Paket erzeugt ein anklickbares Verzeichnis in der PDF-Datei.
\usepackage{hyperref}

%Das Paket wird für die anderthalb-zeiligen Zeilenabstand benötigt
\usepackage{setspace}

%Einrückung eines neuen Absatzes
\setlength{\parindent}{0em}

%Definition der Ränder
\usepackage[paper=a4paper,left=30mm,right=30mm,top=30mm,bottom=30mm]{geometry} 

%Abstand der Fussnoten
\deffootnote{1em}{1em}{\textsuperscript{\thefootnotemark\ }}

%Regeln, bis zu welcher Tiefe (section,subsection,subsubsection) Überschriften angezeigt werden sollen (Anzeige der Überschriften im Verzeichnis / Anzeige der Nummerierung)
\setcounter{tocdepth}{3}
\setcounter{secnumdepth}{3}



\begin{document}

%Beginn der Titelseite
\begin{titlepage}
\begin{small}
\vfill {HTWK Leipzig\\ 
Fachbereich IMN \\ 
Wintersemester 2012/2013}
\end{small}


\begin{center}
\begin{Large}
\vfill {\textsf{\textbf{
Die Möglichkeiten des System-Managements der Firewall-Distribution pfSense\\
}}}
\end{Large}
Beleg im Fach Netzwerk- und System-Management
\end{center}

\begin{small}
\vfill Marcel Kirbst \\ Sieglitz 39 \\  06618 Molau \\
marcel.kirbst@stud.htwk-leipzig.de\\
\today
\end{small}

\end{titlepage}
%Ende der Titelseite


%Inhaltsverzeichnis (aktualisiert sich erst nach dem zweiten Setzen)
\tableofcontents
\thispagestyle{empty}

%Beginn einer neuen Seite
\clearpage

%Anderthalbzeiliger Zeilenabstand ab hier
\onehalfspacing

\pagestyle{plain}


\section{Einleitung}
Dieser Beleg befasst sich der Vorstellung der Routerdistribution pfSense.
Im Vergleich zu den unzähligen anderen, existierenden Routerdistributionen
zeichnet sich pfSense durch seinen hohen Funktionsumfang aus, der beispielsweise
auch Funktionen zur Sicherstellung von Redundanz und Ausfallsicherheit umfasst,
wie sie sonst nur bei preisintensiven proprietären Lösungen kommerzieller
Anbieter verfügbar sind.


Nachdem grundlegende Begriffe erläutert wurden, soll kurz auf die
Entwicklungsgeschichte und Vorzüge des Betriebssystems FreeBSD eingegangen
werden, welches die Grundlage für pfSense bildet. Im folgenden soll ein kurzer
Überblick über den Funktionsumfang von pfSense gegeben werden, für sich genommen
und im Vergleich zu anderen Rouuterdistributionen. Abschließend sollen
beispielhaft drei Konfigurationen vorgestellt werden um die Vielseitigkeit und
Leistungsfähigkeit von pfSense zu demonstrieren.


\section{Routerdistributionen - Besonderheiten und Merkmale im Allgemeinen}

\subsection{Begrifflichkeiten im Zusammenhang mit Routerdistributionen}
Routerdistributionen sind auf einen speziellen Einsatzzweck hin optimierte
Betriebssysteme.

Unter dem Begriff Betriebssystem fasst man eine Menge von Software zusammen, die
auf einem Rechnersystem nach dem Start zur Ausführung kommt, die Ressourcen
dieses Rechnersystems verwaltet und es ermöglicht weitere Anwenungsprogramme zu
starten.

Routerdistributionen werden in der Regel so konzipiert und entwickelt,
um direkt auf einem Rechnersystem installiert zu werden und über alle Ressourcen
dieses Rechnersystems zu verfügen. Dieser Annahme kommt eine besonders hohe
Bedeutung zu, da die beiden Schwerpunkte einer Routerdistribution Sicherheit und
Stabilität darstellen. Andere Merkmale wie zum Beispiel m\"oglichst hoher
Funktionsumfang besitzen dem gegen\"uber niedrigere Priorit\"at, wobei jedoch
verschiedene Routerdistributionen die einzelnen Merkmale im
Detail unterschiedlich stark priorisieren.

Ein Router ist ein Netzwerkgerät, das mit mindestens zwei
Netzwerkschnittstellen ausgestattet ist und den Netzwerkverkehr zwischen den
betreffenden Netzwerken, unter Beachtung eines vorgegebenen Regelwerkes,
vermittelt.

\subsection{Merkmale von Routerdistributionen}
Routerdistributionen werden in der Regel nicht von Grund auf entwickelt sondern
basieren auf einem modifizierten Betriebssystem. Trotz dieser Modifikationen
unterliegen die verschiedenen Routerdistributionen somit mehr oder weniger
stark den Merkmalen, Besonderheiten und Einschr\"ankungen des jeweils zu Grunde
liegenden Betriebssystems. Somit ist das zu Grunde liegende Betreibssystem ein
erstes wichtiges Unterscheidungskriterium f\"ur Routerdistributionen.

Ein weiteres wichtiges Unterscheidungskriterium stellt die Art der Entwicklung
und Lizenzierung dar. Es existieren kommerzielle Produkte genauso wie
quelloffene Produkte. Da kommerzielle Produkte in den allermeisten F\"alle
jedoch nicht im Quellcode verf\"ugbar und somit schwer an spezielle
Bed\"urfnisse anzupassen sind und au\ss{}erdem oft betr\"achtliche Lizenzkosten
verursachen, soll dieser Typus von Routerdistributionen in dieser Arbeit
au\ss{}en vor bleiben. 
 

\subsection{Konkrete Routerdistributionen im Vergleich}
Hier beginnt der zweite Unterabschnitt des ersten Hauptteils.

\section{Ausgew\"ahlte Anwendungsf\"alle von pfSense}
Hier beginnnt der zweite Hauptteil des Belegs.

\subsection{pfSense als DSL-Router}
Hier beginnt der erste Unterabschnitt des zweiten Hauptteils.

\subsection{pfSense als redundanter Firewall-Cluster}
Hier beginnt der zweite Unterabschnitt des zweiten Hauptteils.


\section{Schluss}
Dies ist der Schlussteil.
%Beginn einer neuen Seite
\clearpage

\section{Glossar}
\begin{description}
 \item[Router] Ein Rechnersystem mit mindestens zwei Netzwerkschnittstellen,
das Netzwerkschverkehr zwischen diesen Netzwerkschnittstellen nach einem
Regelwerk vermittelt und weiterleitet.
 \item[Routerdistribution] Eine spezielle Art von Betriebssystem, deren
Hauptaugenmerk bei der Konzeption und Entwicklung darauf liegt
Router-Funktionen sicher und stabil auszuf\"uhren
\end{description}
\clearpage

\section{Literaturverzeichnis}

Musterfrau, Renate: Muster. Frankfurt 2003.


Mustermann, Helmut: Noch ein Muster. Mit einer Einleitung hrsg. von Frank Muster. Frankfurt 2003.


\end{document}
%-------------------
%Hier endet der Text deiner Hausarbeit
%-------------------
